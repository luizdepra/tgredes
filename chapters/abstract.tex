\chapter*{Resumo}

O gerenciamento de redes foi inicialmente impulsionado pela necessidade de monitoramento e controle dos dispositivos que comp�e as redes. Atualmente, com a r�pida evolu��o de aplica��es distribu�das, o controle eficiente de recursos computacionais em uma rede tornou-se essencial. Essa efici�ncia est� diretamente ligada � forma de armazenamento e pesquisa das informa��es relativas ao monitoramento. Este trabalho tem como objetivo a compara��o experimental de duas maneiras distintas para tratar o problema de tratamento de dados gerenciais. O primeiro m�todo consiste na utiliza��o de estruturas de dados 'multidimensionais' que s�o adequadas a esse tipo de dado. No caso, s�o utilizadas as �rvores \textit{K-D} para monitoramento \textit{online} e �rvores \textit{K-D-B} para monitoramento \textit{offline}. O segundo m�todo consiste no uso de um meio tradicional, atrav�s do MySQL, um SGBD muito popular. Atrav�s dos resultados obtidos experimentalmente, existe um ganho positivo com a utiliza��o de estruturas pr�prias para o tratamento de dados de gerencia de redes.
