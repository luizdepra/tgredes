\chapter{Conclus�o}

  Neste trabalho comparamos o desempenho de busca e inser��o de registros com m�ltiplas chaves em �rvores multidimensionais, \textit{K-D} e \textit{K-D-B}, e um banco de dados relacional, o MySQL. Os resultados obtidos atrav�s dos  experimentos em mem�ria prim�ria apontaram que a �rvore \textit{K-D} teve resultados melhores que o MySQL. Na compara��o das estruturas em mem�ria secund�ria, os testes iniciais mostraram que a �rvore \textit{K-D-B} foi mais eficiente que o MySQL. Mas, com alguns ajustes no banco de dados, a diferen�a entre os resultados foi reduzida. Os resultados nos mostraram que, as estruturas de dados podem ser mais eficientes que sistemas de armazenamento complexos.

  Investigando as estruturas usadas pelo Mysql para organizar os �ndices, foi encontrado refer�ncia a �rvore-\textit{R}, que � utilizada para armazenar �ndices espaciais. A �rvore-\textit{R} � uma estrutura de �rvore balanceada parecida com a \textit{B}, em que cada n� da �rvore � associada uma caixa de fronteira semelhante a �rvore \textit{KD}. Essa estrutura � bastante utilizada para armazenar �ndices de estruturas multidimensionais. 
  
  Os experimentos realizados foram simples. Novos experimentos mais detalhados poderiam mostrar de uma forma mais clara os resultados, como por exemplo poderia ser utilizada as ferramentas para armazenamento de dados multidimensionais disponibilizada pelo SGBD. Al�m disso, aprofundar o estudo das ferramentas envolvidas melhoraria o entendimento do seu funcionamento e a interpreta��o dos resultados.

  Trabalhos futuros incluem o desenvolvimento de uma interface Web para a ferramenta, permitindo a configurac�o, monitora��o e consulta multidimensional. Tamb�m est� previsto o acoplamento de um m�dulo para gerac�o de gr�ficos dos recursos monitorados. Por fim, a integrac�o da ferramenta a um sistema de grade comercial com acordos de n�vel de servi�o � tamb�m objetivo do trabalho.