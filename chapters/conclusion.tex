\chapter{Conclus�o}

  Neste trabalho comparamos o desempenho de busca e inser��o de registros com m�ltiplas chaves em �rvores multidimensionais, \textit{K-D} e \textit{K-D-B}, e banco de dados relacional, \textit{MySQL}. Realizamos experimentos de inser��o e com os mesmos dados gerados aleatoriamente para cada ferramenta empregada. Na compara��o das estruturas em mem�ria secund�ria, os testes iniciais mostraram que a �rvore \textit{K-D-B} foi mais eficiente que o \textit{MySQL}. Mas com alguns ajustes no banco de dados a diferen�a entre o resultado foi reduzido. Os resultados nos mostraram que, em alguns casos, estruturas pr�prias podem ser mais eficientes que sistemas de armazenamento complexos. Mas a falta de rapidez pode ser compensada com mais seguran�a contra falhas. Ent�o podemos concluir que a escolha certa depende do tipo de aplica��o me que o sistema de armazenamento ser� utilizado. 
