\chapter{Conclus�o}

  Neste trabalho comparamos o desempenho de busca e inser��o de registros com m�ltiplas chaves em �rvores multidimensionais, \textit{K-D} e \textit{K-D-B}, e banco de dados relacional, MySQL. Os resultados obtidos atrav�s dos  experimentos em mem�ria prim�ria apontaram que a �rvore \textit{K-D} teve resultados melhores que o MySQL. Na compara��o das estruturas em mem�ria secund�ria, os testes iniciais mostraram que a �rvore \textit{K-D-B} foi mais eficiente que o MySQL. Mas com alguns ajustes no banco de dados a diferen�a entre o resultado foi reduzido. Os resultados nos mostraram que, em alguns casos, estruturas pr�prias podem ser mais eficientes que sistemas de armazenamento complexos.
  
  Os experimentos realizados foram simples. Novos experimentos mais detalhados poderiam mostrar de uma forma mais clara os resultados. Al�m disso, aprofundar o estudo das ferramentas envolvidas melhoraria o entendimento do seu funcionamento e a interpreta��o dos resultados.