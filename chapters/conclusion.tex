\chapter{Conclus�o}

	Neste trabalho comparamos o desempenho de busca e inser��o de registros com
	m�ltiplas chaves em �rvores multidimensionais, a �rvore \textit{K-D} e a
	�rvore \textit{K-D-B}, e um banco de dados relacional, o MySQL. Os
	resultados obtidos atrav�s dos  experimentos em mem�ria prim�ria apontaram
	que a �rvore \textit{K-D} teve, como esperado, resultados melhores que o
	MySQL. Na compara��o das estruturas em mem�ria secund�ria, os testes
	iniciais mostraram que a �rvore \textit{K-D-B} foi mais eficiente que o
	MySQL. Por�m, com alguns ajustes no banco de dados, a diferen�a entre os
	resultados foi reduzida. 

	Trabalhos futuros podem incluir o desenvolvimento de uma interface web para
	a ferramenta, que permita a configurac�o, monitora��o e consulta
	multidimensional.  Al�m disso, pode ser acoplado um m�dulo para gerac�o de
	gr�ficos dos recursos monitorados. Investigando as estruturas usadas pelo
	MySQL para organizar os �ndices, foi encontrada como refer�ncia a �rvore
	\textit{R} \cite{guttman}, que � utilizada para armazenar �ndices espaciais.
	A �rvore \textit{R} � uma estrutura de �rvore balanceada semelhante � �rvore
	\textit{B}, em que cada n� da �rvore � associado a uma faixa de fronteira
	semelhante a �rvore \textit{K-D}. Essa estrutura � bastante utilizada para
	armazenar �ndices de estruturas multidimensionais. Novos experimentos
	poderiam ser realizados com a �rvore \textit{R}. Por fim, a integrac�o da
	ferramenta a um sistema de grade comercial com acordos de n�vel de servi�o
	tamb�m pode ser tema para trabalhos futuros.
