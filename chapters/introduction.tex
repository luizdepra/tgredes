\chapter{Introdu��o}

	A import�ncia do gerenciamento de recursos computacionais tem crescido em ambiente de rede, principalmente com o surgimento de novas tecnologias que recursos distribu�dos. Para o bom funcionamento do gerenciamento � necess�rio que o mesmo apresente um bom desempenho, o que tem dependido principalmente do tipo de armazenamento de dados. Escolher a tecnologia de armazenamento mais �gil resultar� em um ganho no desempenho total do sistema. A escolha mais intuitiva seria um banco de dados relacional, por�m � poss�vel utilizar outras estruturas como uma �rvore.
	
	O objetivo deste trabalho � comparar o desempenho da inser��o e pesquisa em uma banco de dados relacional, o \textit{MySQL}, com duas �rvores de pesquisa multidimensional, \textit{K-D} e \textit{K-D-B}.   
