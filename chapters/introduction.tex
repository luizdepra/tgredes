\chapter{Introdu��o}

  As redes de computadores e aplica��es distribu�das t�m se tornado ferramentas indispensaveis para as organiza��es. Por esse motivo � fundamental que o servi�os de rede tenham desempenho e taxa de falhas aceitaveis. Garantir a qualidade dos servi�os de rede � tarefa do gerenciamento de rede \cite{NetworkManagement}, que � definido como o processo de controle de uma complexa rede de dados visando maximizar sua efici�ncia e produtividade.
	
	A gerencia de rede � dividida em quatro partes: elementos gerenciados, esta��es de ger�ncia, protocolos de ger�ncia e informa��es de ger�ncia \cite{GerenciaRede}. Os elementos gerenciados s�o os componentes da redes que precisam operar adequadamente para que toda a rede ofere�a os servi�oes para quais foi projetada. Cada um dos elementos gerenciados possui um software especial chamado agente, que permite que os elementos gerenciados sejam controlados e a monitorados remotamente. As esta��es de ger�ncia s�o responsaveis controle e monitoramento dos elementos gerenciados. Tamb�m munidas de um software especial, chamado gerente, as esta��es... continuar
  
  Paragrafo de gerencia: componentes basicos, SLM, SLA 
  
  O objetivo deste trabalho � comparar o desempenho do uso de �rvores multidimensionais com o uso de bancos de dados relacionais, em especial o MySQL, no armazenamento de dados obtidos a partir da monitora��o de um rede. Para isso, foram realizados experimentos de busca e inser��o em �rvore \textit{K-D}, �rvore \textit{K-D-B} e banco de dados MySQL.

  Paragrafo sobre os resultados

  Este trabalho est� organizado da seguinte forma. O cap�tulo 2 apresenta a pesquisa multidimensional incluindo defini��es b�sicas e as estruturas de dados utilizadas no trabalho. O cap�tulo 3 descreve os experimentos realizados, e resultados obtidos. O trabalho finaliza no cap�tulo 4 com a conclus�o.  
