\chapter{Introdu��o}

  A import�ncia do gerenciamento de redes tem crescido com o aumento da complexidade destes sistemas e do fato de que t�m se tornado cr�tico para as organiza��es. O bom funcionamento do gerenciamento implica que o mesmo apresente bom desempenho. A monitora��o concorrente de m�ltiplos par�metros que representam o n�vel de servi�o do sistema ...
  
  O objetivo deste trabalho � comparar o desempenho da inser��o e busca com muitas chaves, simulando dados de gerencia de rede. A compara��o feita � entre um banco de dados relacional, o \textit{MySQL}, e duas �rvores de pesquisa multidimensional, \textit{K-D} e \textit{K-D-B}. ...

  Este trabalho est� organizado da seguinte forma. O cap�tulo 2 apresenta a pesquisa multidimensional com algumas defini��es b�sicas e de estruturas de dados utilizadas no trabalho. O cap�tulo 3 trata dos experimentos realizados mostrando os procedimentos utilizados e resultados obtidos. Finalizando com a conclus�o do trabalho.  
