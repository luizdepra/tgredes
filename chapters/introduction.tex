\chapter{Introdu��o}

  As redes de computadores e aplica��es distribu�das t�m se tornado ferramentas indispensaveis para as organiza��es. Por isso � fundamental que o servi�os mantenham o bom desempenho e uma taxa de falhas aceitavel. Essa tarefa � atribuida ao gerenciamento de rede \cite{NetworkManagement}, que � definico como o processo de controle de uma complexa rede de dados visando maximizar sua efici�ncia e produtividade.
  
  Paragrafo de gerencia: componentes basicos, SLM, SLA 
  
  O objetivo deste trabalho � comparar o desempenho do uso de �rvores multidimensionais com o uso de bancos de dados relacionais, em especial o MySQL, no armazenamento de dados obtidos a partir da monitora��o de um rede. Para isso, foram realizados experimentos de busca e inser��o em �rvore \textit{K-D}, �rvore \textit{K-D-B} e banco de dados MySQL.

  Paragrafo sobre os resultados

  Este trabalho est� organizado da seguinte forma. O cap�tulo 2 apresenta a pesquisa multidimensional incluindo defini��es b�sicas e as estruturas de dados utilizadas no trabalho. O cap�tulo 3 descreve os experimentos realizados, e resultados obtidos. O trabalho finaliza no cap�tulo 4 com a conclus�o.  
