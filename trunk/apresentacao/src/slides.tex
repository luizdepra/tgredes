\documentclass{beamer}

%\usepackage{beamerthemesplit}
\usepackage[brazil]{babel}
%\usepackage[utf8]{inputenc}
%\usepackage[T2A]{fontenc}
%descomente para ISO e comente as duas de cima
\usepackage[latin1]{inputenc}
%\usetheme{Frankfurt}
%\usetheme{Singapore}
\usetheme{Boadilla}
%\usetheme{Berkeley}
\useinnertheme{rounded}
\usefonttheme{structurebold}
\usepackage{latexsym}
\usepackage{verbatim}
\usepackage{graphicx}
\usepackage{epsfig}

%transparenz
\setbeamercovered{
still covered={\opaqueness<1>{15}\opaqueness<2>{10}
\opaqueness<3>{5}\opaqueness<4->{2}},
again covered={\opaqueness<1->{15}}
}
\setbeamertemplate{navigation symbols}{}


\title[Trabalho de Gradua��o]{ Experimentos de Pesquisa Multidimensional em �rvores \textit{K-D} e \textit{K-D-B} versus Banco de Dados Relacional }
\author[UFPR]{Luiz Francisco Artigas de Pr�\\ Renato Yamazaki \\ Saulo Quinteiro dos Santos}
\institute[DINF]{Universidade Federal do Paran� \\ \today}
\date[\today]{Orientador: Elias Proc�pio Duarte Jr. \\ Co-orientador: Andreas Kiefer}

\begin{document}

\frame{\titlepage}

%\section[Roteiro]{}
%\frame{\tableofcontents}

\AtBeginSection[]
{
    \begin{frame}
    	\frametitle{Roteiro}
	\tableofcontents[currentsection]
    \end{frame}
}

\section{Motiva��o}

\begin{frame}
	\frametitle{Porque Comparar o MySQL com �rvores Multidimensionais?}
	\begin{itemize}
		\item Com a evolu��o das aplica��es distribu�das, a ger�ncia eficiente dos recursos computacionais tornou-se essencial.
		\item A ger�ncia de redes consiste no controle e monitora��o dos elementos de uma rede, para atingir uma melhor utiliza��o dos recursos e assegurar um certo n�vel de servi�o.
		\item A motiva��o � estudar a efici�ncia no armazenamento e pesquisa de informa��es gerenciais, usando abordagens diferentes.
	\end{itemize}
\end{frame}



\section{Pesquisa Multidimensional}

\begin{frame}
	\frametitle{Defini��es preliminares}
	\begin{itemize}
		\item O objetivo da pesquisa � recuperar informa��o a partir de uma grande massa de dados.
		\item Esses dados s�o divididos em registros, e cada registro possui uma chave. Essa chave � usada na pesquisa para encontrar o registro.
	\end{itemize}
\end{frame}

\begin{frame}
	\frametitle{O que � pesquisa multidimensional?}
	\begin{itemize}
		\item Um algoritmo de pesquisa recebe uma chave de pesquisa como entrada, e retorna os registros com chaves iguais � chave de pesquisa. 
		\item Quando m�ltiplas chaves s�o usadas, a pesquisa � dita multidimensional. 
		\item A pesquisa multidimensional pode ser do tipo exata, parcial ou dentro de um intervalo.
	\end{itemize}
\end{frame}

\begin{frame}
	\frametitle{Estruturas de dados multidimensionais}
	\begin{itemize}
		\item Um registro multidimensional possui m�ltiplas chaves. 
		\item As estruturas de dados multidimensionais indexam e d�o suporte a registros multidimensionais.
		\item A pesquisa multidimensional � eficiente - logar�tmica - se feita com estruturas de dados multidimensionais.
	\end{itemize}
\end{frame}



\section{�rvores Multidimensionais}

%%%%%%%%%%%%%%%%%%%%%%%%%%%%%%%%%%%%%%%%%%%%%%%%%%%%%%%%%%%%%%%%%%%%%%%%%%%%%%%%
\begin{frame}
	\frametitle{�rvore \textit{K-D}}
	\begin{itemize}
		\item Semelhante � �rvore bin�ria, por�m possibilita a pesquisa por m�ltiplos atributos.
		\item Na �rvore \textit{K-D} os registros s�o definidos por \textit{K} chaves.
		\item A chave que determina a sub-�rvore que ser� acessada, varia a cada n�vel.
		\item No n�vel $L$ a chave $L\ mod\ (K+1)$ � utilizada.
	\end{itemize}
\end{frame}

\begin{frame}
	\frametitle{�rvore \textit{K-D}}
	\begin{figure}[h]
	\center
	\subfigure{\includegraphics[width=5cm]{../images/algorithm/kd_plano.png}}
	\qquad
	\subfigure{\includegraphics[width=5cm]{../images/algorithm/kd_passo4.png}}
	\end{figure}
\end{frame}

\begin{frame}
	\frametitle{�rvore \textit{K-D}: Opera��es}
	\textbf{Pesquisa}
	\begin{itemize}
		\item Pode ser classificada em pesquisa em exata, parcial ou por intervalo.
		\item Percorre a estrutura usando os crit�rios escolhidos (de acordo com o tipo de pesquisa).		
	\end{itemize}
	\textbf{Inser��o}
	\begin{itemize}
		\item Percorre a estrutura usando da pesquisa exata.
		\item Ao encontrar um n� folha, cria um n� com os novos dados e se torna filho do n� folha encontrado.
	\end{itemize}
\end{frame}

\begin{frame}
	\frametitle{�rvore \textit{K-D}: Inser��o}
	\begin{figure}[h]
	\center
	\includegraphics[width=10cm]{../images/algorithm/kd_passo1.png}
	\end{figure}
\end{frame}

\begin{frame}
	\frametitle{�rvore \textit{K-D}: Inser��o}
	\begin{figure}[h]
	\center
	\includegraphics[width=10cm]{../images/algorithm/kd_passo2.png}
	\end{figure}
\end{frame}

\begin{frame}
	\frametitle{�rvore \textit{K-D}: Inser��o}
	\begin{figure}[h]
	\center
	\includegraphics[width=10cm]{../images/algorithm/kd_passo3.png}
	\end{figure}
\end{frame}

\begin{frame}
	\frametitle{�rvore \textit{K-D}: Inser��o}
	\begin{figure}[h]
	\center
	\includegraphics[width=10cm]{../images/algorithm/kd_passo4.png}
	\end{figure}
\end{frame}

\begin{frame}
	\frametitle{�rvore \textit{K-D}: Inser��o}
	\begin{figure}[h]
	\center
	\includegraphics[width=10cm]{../images/algorithm/kd_passo5.png}
	\end{figure}
\end{frame}
%%%%%%%%%%%%%%%%%%%%%%%%%%%%%%%%%%%%%%%%%%%%%%%%%%%%%%%%%%%%%%%%%%%%%%%%%%%%%%%%
\begin{frame}
	\frametitle{�rvore \textit{K-D Adaptativa}}
	\begin{itemize}
		\item Varia��o da �rvore \textit{K-D}, resolve o problema do desbalanceamento utilizando o algoritmo de Hoare.
		\item O algoritmo encontra a mediana sobre os pontos da parti��o, para que a nova parti��o tenha o mesmo n�mero de pontos.
		\item Funciona melhor se os dados s�o conhecidos a priori.
	\end{itemize}
\end{frame}

\begin{frame}
	\frametitle{�rvore \textit{K-D Adaptativa}}
	\begin{figure}[h]
	\center
	\subfigure{\includegraphics[width=5cm]{../images/algorithm/kda_plano.png}}
	\qquad
	\subfigure{\includegraphics[width=5cm]{../images/algorithm/kd_1.png}}
	\end{figure}
\end{frame}

\begin{frame}
	\frametitle{�rvore \textit{K-D Adaptativa}: Opera��es}
	\textbf{Pesquisa}
	\begin{itemize}
		\item Semelhante ao algoritmo de pesquisa da �rvore \textit{K-D}. Possui os mesmos tipos de pesquisa: exata, parcial e por intervalo.
		\item Por�m, utiliza a mediana entre os pontos do plano atual, ao inv�s da chave do n� atual.
	\end{itemize}
	\textbf{Inser��o}
	\begin{itemize}
		\item Segue o mesmo racioc�nio da pesquisa, utilizando a pesquisa exata.
		\item Ao encontrar um n� folha, cria um n� com os novos dados e se torna filho do n� folha encontrado.
	\end{itemize}
\end{frame}

\begin{frame}
	\frametitle{�rvore \textit{K-D Adaptativa}: Inser��o}
	\begin{figure}[h]
	\center
	\includegraphics[width=10cm]{../images/algorithm/kda_1.png}
	\end{figure}
\end{frame}

\begin{frame}
	\frametitle{�rvore \textit{K-D Adaptativa}: Inser��o}
	\begin{figure}[h]
	\center
	\includegraphics[width=10cm]{../images/algorithm/kda_2.png}
	\end{figure}
\end{frame}
%%%%%%%%%%%%%%%%%%%%%%%%%%%%%%%%%%%%%%%%%%%%%%%%%%%%%%%%%%%%%%%%%%%%%%%%%%%%%%%%
\begin{frame}
	\frametitle{�rvore \textit{K-D-B}}
	\begin{itemize}
		\item Possui caracter�sticas de �rvore \textit{K-D adaptativa} e �rvore \textit{B}.
		\item Os n�s internos s�o apontadores e representam determinada regi�o. S�o chamados de \textit{p�ginas de regi�es} e cont�m o endere�o para o n� filho.
		\item Os dados (pontos) s�o armazenados somente nas \textit{p�ginas de pontos} (folhas).
		\item As folhas est�o sempre em um mesmo n�vel, e isso garante que a �rvore esteja balanceada.
	\end{itemize}
\end{frame}

\begin{frame}
	\frametitle{�rvore \textit{K-D-B}}
	\begin{figure}[h]
	\center
	\includegraphics[width=10cm]{../images/kdb.jpg}
	\end{figure}
\end{frame}

\begin{frame}
	\frametitle{�rvore \textit{K-D-B}: Opera��es}
	\textbf{Pesquisa}
	\begin{itemize}
		\item Pode ser classificada em pesquisa exata, parcial e por intervalo.
		\item Os algoritmos de pesquisa fazem um caminho sobre a �rvore at� achar a p�gina de folha, usando a divis�o dos planos.
	\end{itemize}
	\textbf{Inser��o}
	\begin{itemize}
		\item Para inserir um ponto $p$ na �rvore, � localizada a p�gina folha apropriada para o ponto. Se existe espa�o na p�gina folha, $p$ � inserido, caso n�o, a p�gina � dividida em duas.
	\end{itemize}
\end{frame}

\section{Banco de Dados Relacional}

\begin{frame}
	\frametitle{Banco de dados Relacional}
	\begin{itemize}
		\item Bancos de dados, s�o conjuntos de registros dispostos em estrutura regular que possibilitam a reorganiza��o dos mesmos e produ��o de informa��o.
		\item Um banco de dados relacional � um conceito abstrato que define
	maneiras de armazenar, manipular e recuperar dados estruturados unicamente
	na forma de tabelas. 
		\item A tabela � um conjunto de dados dispostos em n�mero
	finito de colunas e n�mero ilimitado de linhas (ou tuplas).
	\end{itemize}
\end{frame}


\begin{frame}
	\frametitle{SGBD Sistema Gerenciador de Bando de Dados}
	\begin{itemize}
		\item � um conjunto de programas respons�vel pelo gerencialmento de uma base de dados que disponibiliza uma interface para incluir, alterar ou consultar dados.
		\item Essa interface � constitu�da pelas APIs ou drivers do SGBD, que executam comandos na linguegem SQL
	\end{itemize}
\end{frame}


\begin{frame}
	\frametitle{MySQL}
	\begin{itemize}
		\item O MySQL � um SGBD bastante popular, distribu�do como
	software livre, que utiliza a linguagem SQL como interface.
	\end{itemize}
\end{frame}

\section{Experimentos}

\begin{frame}
	\frametitle{Decis�es de Implementa��o}
	\begin{itemize}
		\item Para a medi��o do tempo foi utilizada a fun��o \textit{clock()} da biblioteca \textit{time.h}.
		\item Arquivos de entrada texto. Cada linha representa um dado multidimensional, e cada coluna uma dimens�o desse dado.
		\item 
	\end{itemize}
\end{frame}



\section{Conclus�o}

\begin{frame}
	\frametitle{Conclus�o}
	\begin{itemize}
		\item �rvore \textit{K-D} obteve resultados melhores que o MySQL. Isso era esperado pois a �rvore \textit{K-D} trabalha somente com mem�ria prim�ria.
		\item �rvore \textit{K-D-B} obteve inicialmente resultados melhores que o MySQL. Com ajustes no MySQL, e utiliza��o de uma forma de inser��o otimizada o resultado foi invertido.
		\item Investiga��es mostraram que o MySQL pode usar a �rvore \textit{R} para organizar os �ndices. A �rvore \textit{R} serve para armazenar �ndices espaciais.
	\end{itemize}
\end{frame}

\section{Trabalhos Futuros}

\begin{frame}
	\frametitle{Trabalhos Futuros}
	\begin{itemize}
		\item a
		\item b
		\item c
	\end{itemize}
\end{frame}


\end{document}
