\chapter*{Resumo}

Os sistemas de gerenciamento de redes t�m como fun��o principal o monitoramento e controle dos dispositivos que comp�e as redes. Atualmente, com a r�pida evolu��o de aplica��es distribu�das, a ger�ncia eficiente de recursos computacionais em uma rede tornou-se essencial. Destacam-se as aplica��es que envolvem computa��o remota e todas aquelas que apresentam requisitos de n�vel de servi�o que devem ser garantidos. O monitoramento cont�nuo de m�ltiplas caracter�sticas de recursos e servi�os resulta em quantidades gigantescas de dados. A efici�ncia da ger�ncia est� diretamente ligada � forma de armazenamento e pesquisa desses dados. Este trabalho tem como objetivo a compara��o experimental de duas estrat�gias distintas para a manipula��o desses dados. O primeiro m�todo consiste na utiliza��o de estruturas de dados multidimensionais, em particular s�o utilizadas as �rvores \textit{K-D} para monitoramento \textit{online} e �rvores \textit{K-D-B} para monitoramento \textit{offline}. O segundo m�todo consiste no uso de um meio tradicional, atrav�s de um banco de dados relacional, o MySQL. Os experimentos realizados foram �rvore \textit{K-D} vs MySQL e �rvore \textit{K-D-B} vs MySQL para busca e inser��o.
