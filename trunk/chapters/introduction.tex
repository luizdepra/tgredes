\chapter{Introdu��o}

  Definimos gerenciamento de rede \cite{NetworkManagement} como o processo de controle de uma complexa rede de dados visando maximizar sua efici�ncia e produtividade. O gerenciamento de redes t�m sido um fator cr�tico para as organiza��es devido ao crescimento da complexidade dos sistemas.
  
  O armazenamento dos par�metros de monitora��o pode ser feito de varias maneiras. Algumas formas vi�veis de armazenamento consistem em utilizar bancos de dados relacionais �rvores multidimensionais. O principal diferencial entre essas duas abordagens s�o a forma em que s�o feitas as pesquisas e inser��es de dados. Em bancos de dados relacionais a inser��o pode ser feita usando �ndices, que melhoram o tempo de busca mas pioram o de inser��o. A busca depende da estrutura que o banco de dados e do tipo de pesquisas que ser�o feitas. �rvores \textit{B+} ou fun��es Hash s�o as estruturas mais comuns. Em �rvores multidimensionais os registros s�o organizados de acordo com o valor de cada par�metro e n�vel da �rvore. Tanto a busca quanto a inser��o ocorrem de forma semelhante, percorrendo a estrutura.
  
  O objetivo deste trabalho � comparar o desempenho da inser��o e busca de registros com m�ltiplas chaves, correspondentes a dados obtidos a partir da monitora��o de uma rede. A compara��o feita � entre um banco de dados relacional, o \textit{MySQL}, e duas �rvores de pesquisa multidimensional, \textit{K-D} e \textit{K-D-B}.

  Este trabalho est� organizado da seguinte forma. O cap�tulo 2 apresenta a pesquisa multidimensional com algumas defini��es b�sicas e de estruturas de dados utilizadas no trabalho. O cap�tulo 3 trata dos experimentos realizados mostrando os procedimentos utilizados e resultados obtidos. Finalizando com a conclus�o do trabalho.  
