\chapter{Introdução}

	As redes de computadores e aplicações distribuídas têm se tornado ferramentas indispensáveis para as organizações e pessoas. Essas aplicações têm requisitos específicos de desempenho que devem ser atendidos tanto na infra-estrutura da rede quanto nos pontos de origem e destino. Para garantir que os serviços estejam sendo providos seguindo as especificações das aplicações, é necessário o gerenciamento dos recursos de cada rede conectada conectada que oferece os serviços. 
	
	A gerência de redes \cite{NetworkManagement} consiste no controle e monitoração dos elementos de uma rede, quer sejam físicos ou lógicos, para atingir uma melhor utilização dos seus recursos e assegurar um certo nível de serviço. Os níveis de serviço são definidos como contratos de níveis de serviços (SLA - \textit{Service Level Agreements}). Esses contratos são firmados entre provedores e clientes de serviços. SLAs explicitam formalmente o que um provedor de serviço irá entregar aos seus clientes. Além disso, os SLAs definem os critérios de aceitabilidade de um serviço, as formas de verificar se os critérios de aceitabilidade estão sendo assegurados, além de incluir os compromissos tanto do provedor como do cliente.

	O SNMP (\textit{Simple Network Management Protocol}) é o protocolo padrão de gerência de redes TCP/IP. A arquitetura geral dos sistemas de gerência de redes apresenta 4 componentes básicos: elementos gerenciados, estações de gerência, protocolos de gerência e informações de gerência. Os elementos de gerenciados podem ser computadores, protocolos, aplicativos ou qualquer outra entidade que disponibilize informações à rede - podendo assim serem controlados e monitorados a partir de uma ou mais estações de gerência.



	O objetivo deste trabalho é comparar o desempenho do uso de árvores multidimensionais com o uso de bancos de dados relacionais, em especial o MySQL, no armazenamento de dados obtidos a partir da monitoração de uma rede. Para isso, foram realizados experimentos de busca e inserção em árvore \textit{K-D}, árvore \textit{K-D-B} e banco de dados MySQL.

	Paragrafo sobre os resultados

	Este trabalho está organizado da seguinte forma. O capítulo 2 apresenta a pesquisa multidimensional incluindo definições básicas e as estruturas de dados utilizadas no trabalho. O capítulo 3 descreve os experimentos realizados, e resultados obtidos. O trabalho finaliza no capítulo 4 com a conclusão.  
